\chapter*{Niveau 2}
\addcontentsline{toc}{chapter}{Niveau 2}
Du har lært at brygge de lidt mere avancerede drikke. Du kan se flere muligheder i mange af de urter du kender, og det er nu muligt for dig at påvirke spillet mere end før.\\
\begin{table}[H]
    \centering
    \begin{tabular}{|p{0.50\textwidth}|p{0.25\textwidth}|}
    \rowcolor{cerulean!80}\hline
        Evne navn & Pris i XP \\\hline
         Alkymi Niv. 2 & 2 \\\hline
         Læse/Skrive Magi & 1 \\\hline
         Personlig Have Niv. 1 & 2 \\
         \hline
    \end{tabular}
\end{table}
\section*{Evne beskrivelse}
\addcontentsline{toc}{section}{Evne beskrivelse}

\input{mainstuff/AlkymiNiv2}

\subsection*{Læse/Skrive Magi}
\addcontentsline{toc}{subsection}{Læse/Skrive Magi}
Du kan læse og skrive skrift, der er skrevet på Magisk.\\
\begin{figure}[H]
    \centering
    \includegraphics[width=1\textwidth]{setup/Alfabeter/Magisk alfabet.pdf}
    \caption{Magisk alfabet}
\end{figure}

\subsection*{Personlig Have Niv. 1}
\addcontentsline{toc}{subsection}{Personlig Have Niv. 1}
Som en del af din spilgang kan du vælge et område som skal være din have. Vi foreslår at dette område er afmærket så det er tydeligt.\\
I løbet af en spilgang kan du vælge at plante en urt i din have. Dette betyder at du mister en urt, men ved starten af næste spilgang vil du få 3 af denne urt tilbage.\\
Vi vil meget gerne opfordre spillere til at søge hjælp fra andre til at 'velsigne', 'luge' eller 'gøde' deres have så dette kan give godt spil.\\
\textit{\textbf{Eksempel:} Johannes har en have, hvor han planter Slangefrø. Han mister derfor den pose med Slangefrø, han har, og som en del af spillet søger han hjælp med en druide, og to krigere. Druiden velsigner jorden, mens krigerne hjælper ham med at fjerne ukrudt. Dette bliver godkendt af en arrangør, men han nævner det også ved næste indcheck. Derfor får han udleveret 5 Slangefrø, selvom han normalt kun ville blive givet 3.}
\\
